\section*{ACKNOWLEDGEMENTS}\label{sec:acknow}
\addcontentsline{toc}{section}{ACKNOWLEDGEMENTS}

\textbf{Comentario de Rudik}\\
\indent Fue un proyecto que realmente hizo darme cuenta de varias cosas: (1) Eventualmente, en algún momento me gustaría dedicarme a realizar investigación, ya que es interesante documentarse, experimentar y presentar los resultados. (Como lo del paralaje negativo, que me hizo percatarme que incluso astrónomos de alto calibre, tenían los mismos problemas que yo y que existían algo equivale a \textit{StackOverFlow} pero para astronomía para resolver dudas). (2) Que en el futuro, la astronomía podría ser un camino bastante viable para mí (en algún momento de mi vida quizás me lo plantee más seriamente). (3) Que aunque este semestre haya sido mi último semestre en física, Introducción a la Astronomía definitivamente fue el mejor curso de física que tomé en estos dos últimos años (aunque siento que no me esforcé al 100\%, está pandemia me ha afectado demasiado) que me hizo replantearme si me decisión es la correcta. (4) Que realmente, me habría gustado que este proyecto hubiese tenido programación involucrada y análisis estadísticos; ya que en los papers en los que consulté, parece ser que la programación y la estadística son de las cosas más esenciales y que hacen que sean más enriquecedores.     \newline
\newline
\indent\textbf{Comentario de Baptiste}\\
Ese proyecto es una forma muy interesante de culminar el curso de Introducción a la Astronomía, al menos a nivel personal. Mi rama favorita de la física es la parte teórica, a contrario de Kristhell, quien es una física experimental de corazón. Cuando entré a la Universidad, la astrofísica era mi motivación. Luego de haber llevado un curso de introducción a la física de partículas, esta área se volvió mi favorita. En el momento de llevar el curso de Introducción a la Astronomía, mi visión fue de nuevo cuestionada. El curso me pareció increíble y es fascinante conocer acerca de la vida de los astrofísicos experimentales. Este proyecto, sin embargo, me volvió a asegurar que la parte experimental no me gusta. Analizar imágenes y describir objetos me parece tedioso. Si un día me encamino hacia la Astrofísica, no será en una perspectiva experimental.
