\section*{DISCUSIÓN}\label{sec:discusion}
\addcontentsline{toc}{section}{DISCUSIÓN}

La deducción presentada solo presenta confianza acerca de tres datos. En base al paralaje y la magnitud absoluta, solo fue posible encontrar a la distancia del cuerpo estelar, su luminosidad y su flujo. Con el fin de estimar subsecuentes datos, se ha tenido que realizar el supuesto que el objeto estudiado pertenece a la secuencia principal de estrellas. Así es como se pudo dar un orden de magnitud para la temperatura y así encontrar el color y el radio de la estrella estudiada. Nótese que el análisis en ningún punto es válido si la estrella llegue a no ser parte de la secuencia principal. Se recomienda, especialmente para el caso de la segunda estrella, estudiar la pertenencia a diferentes clases en el diagrama de Hertzsprung-Russell (HR).\\

Se encontró un sesgo al determinar la distancia de la estrella de la Figura \ref{fig:estrella-rudik} y que se presenta en el Cuadro \ref{tab:estrellas} ya que los datos experimentales presentaron un paralaje negativo. Este es un fenómeno poco probable, debido a la certeza del telescopio GAIA. Sin embargo, es un problema que suele estar presente en las observaciones del telescopio y no es recomendado obviarlo, por las implicaciones que podría tener en una investigación con un requerimiento elevado de precisión, como lo explica \cite{gaia2018gaia}. El artículo científico anteriormente mencionado explicaba ciertas metodologías para corregir el paralaje negativo, que se basaban en métodos complejos a simplemente usar la incertidumbre dada por el archivo de GAIA; por lo cual se decidió usar este método. Esto implica, por la poca precisión del método, que los datos obtenidos para la estrella 1 del Cuadro \ref{fig:estrella-bapt} presentan un sesgo enorme y no podrían ser considerados precisos ni confiables.  \\

Finalmente mencionamos que, luego de revisar los datos reportados por \cite{lopez2019quiescent}, se encontró que la segunda imagen no fue analizada en el filtro que fue mencionado al inicio de la realización de este trabajo. Al revisar el cuadro \ref{tab:estrellas2}, notamos que el filtro de observación para la segunda imagen fue $H$ y no $K_s$. Se recomienda revisar la procedencia de las imágenes antes de procesarlas.